\abstract{
Recent research on the response of stock prices to trading activity revealed
long lasting effects, even across stocks of different companies.
These results imply non-markovian effects in price formation and when trading
many stocks at the same time, in particular trading costs and price
correlations.
How the price response is measured depends on data set and research focus.
However, it is important to clarify, how the details of the price response
definition modify the results. Here, we evaluate different price response
implementations for the Trades and Quotes (TAQ) data set from the NASDAQ stock
market and find that the results are qualitatevily the same for two different
definitions of time scale, but the response can vary by up to a factor of two.
Further, we confirm the dominating contribution of immediate price response
directly after a trade, as we find that delayed responses are suppresed.
Finally, we test the impact of the spread in the price response, detecting that
large spreads have stronger impact.
\PACS{
      {89.65.Gh}{Econophysics} \and
      {89.75.-k}{Complex systems} \and
      {05.10.Gg}{Statistical physics}
} % end of PACS codes
} %end of abstract