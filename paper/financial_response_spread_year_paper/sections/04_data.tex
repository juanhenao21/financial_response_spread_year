\section{Data set}\label{sec:data}

Modern financial markets, are organized as a double continuous auctions. Agents
can place different types of orders to buy or to sell a given number of shares,
roughly categorized as market orders and limit orders.

In this study, we analyzed trades and quotes (TAQ) data from the NASDAQ stock
market. We selected NASDAQ because it is an electronic exchange where stocks
are traded through an automated network of computers instead of a trading
floor, which makes trading more efficient, fast and accurate. Furthermore,
NASDAQ is the second largest stock exchange based on market capitalization
in the world.

In the TAQ data set, there are two data files for each stock. One gives the
list of all successive quotes. Thus, we have the best bid price, best ask
price, available volume and the time stamp accurate to the second. The other
data file is the list of all successive trades, with the traded price, traded
volume and time stamp accurate to the second. Despite the one second accuracy
of the time stamps, in both files more than one quote or trade may be recorded
in the same second.

To analyze the response functions across different stocks in Sects.
\ref{sec:response_functions_imp}, \ref{sec:time_shift} and
\ref{sec:short_long}, we select the six companies with the largest average
market capitalization (AMC) in three economic sectors of the S\&P index in
2008. Table \ref{tab:companies} shows the companies analyzed with their
corresponding symbol and sector, and three average values for a year.

\begin{table*}[htbp]
\begin{threeparttable}
\caption{Analyzed companies.}
\begin{tabular*}{\textwidth}{c @{\extracolsep{\fill}} ccccc}
\toprule
\bf{Company} & \bf{Symbol} & \bf{Sector} & \bf{Quotes}\tnote{1} &
\bf{Trades}\tnote{2} & \bf{Spread}\tnote{3}\tabularnewline
\midrule
Alphabet Inc. & GOOG & Information Technology (IT) & $164489$ & $19029$ &
$0.40\$$\tabularnewline
Mastercard Inc. & MA & Information Technology (IT) & $98909$ & $6977$ &
$0.38\$$\tabularnewline
CME Group Inc. & CME & Financials (F) & $98188$ & $3032$ &
$1.08\$$\tabularnewline
Goldman Sachs Group Inc. & GS & Financials (F) & $160470$ & $26227$ &
$0.11\$$\tabularnewline
Transocean Ltd. & RIG & Energy (E) & $107092$ & $11641$ &
$0.12\$$\tabularnewline
Apache Corp. & APA & Energy (E) & $103074$ & $8889$ & $0.13\$$\tabularnewline
\bottomrule
\end{tabular*}
\label{tab:companies}
\begin{tablenotes}\footnotesize
\item[1] Average number of quotes from 9:40:00 to 15:50:00 New York time during
 2008.
\item[2] Average number of trades from 9:40:00 to 15:50:00 New York time during
 2008.
\item[3] Average spread from 9:40:00 to 15:50:00 New York time during 2008.
\end{tablenotes}
\end{threeparttable}
\end{table*}

To analyze the spread impact in response functions (Sect.
\ref{sec:spread_impact}), we select 524 stocks in the NASDAQ stock market for
the year 2008. The selected stocks are listed in Appendix
\ref{app:spread_impact}.

This contribution specifically addresses methodical aspects related to the
price response functions and spread impact in correlated financial markets. We
carefully check and verify the methods used to evaluate response functions,
spread impact and related quantities. This is important as such observables
quantify the deviation from the largely Markovian behavior of financial
markets. We chose the year 2008 to clarify these methodical aspects. We plan to
extend our results in a future study to different years for a comparison of
price response functions and spread impact.

In order to avoid overnight effects and any artifact due to the opening and
closing of the market, we systematically discard the first ten and the last
ten minutes of trading in a given day
\cite{Bouchaud_2004,large_prices_changes,spread_changes_affect,Wang_2016_cross}.
Therefore, we only consider trades of the same day from 9:40:00 to 15:50:00
New York local time. We will refer to this interval of time as the ``market
time". The year 2008 corresponds to 253 business days.
