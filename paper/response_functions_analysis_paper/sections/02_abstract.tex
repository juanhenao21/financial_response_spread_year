\abstract{
Using Trades and Quotes (TAQ) data from the NASDAQ stock market, we analyzed
the response functions of six companies with the largest average market
capitalization in three economic sectors of the S\&P index in 2008. We used two
time definitions to compute the response functions: trade time scale and
physical time scale. We computed the self-response functions for the six
companies and the cross-response functions for the three sectors. Both
responses increase to a maximum and then slowly decrease. Hence, the trend in
the response functions is eventually reversed. To analyze the influence of the
relative position between trade signs and returns, we added a parameter time
shift $\left( t_{s}\right)$ to the response function expression. For negative
and large values of the time shift the information contained between the trade
signs and returns is lost. We also analyzed the impact of the time lag in the
response functions. We divided the time lag in an immediate and late component
and compute the self- and cross-response. The influence of the immediate time
lag is larger than the late time lag. Finally, we analyzed the spread impact in
the response functions for 530 companies in the NASDAQ stock market. We found
that the impact in the response function is bigger when the spread is large.
\PACS{
      {89.65.Gh}{Econophysics} \and
      {89.75.-k}{Complex systems} \and
      {05.10.Gg}{Statistical physics}
     } % end of PACS codes
} %end of abstract